\documentclass[11pt,a4paper]{article}
\usepackage[left=2cm,text={17cm,24cm},top=3cm]{geometry}
\usepackage[T1]{fontenc}
\usepackage[czech]{babel}
\usepackage[utf8]{inputenc}
\usepackage{url}
\usepackage{graphicx}
\usepackage{pdfpages}
\graphicspath{ {img/} }

\begin{document}

\begin{center}
	\LARGE{Paralelní a distribuované algoritmy -- dokumentace k projektu 1}\\
	\large{Vysoké učení technické v Brně}
	\vspace{0.5cm}

	Petr Stehlík <xstehl14@stud.fit.vutbr.cz>

	\vspace{0.2cm}

	\today

\end{center}

\section{Zadání}

Cílem projektu byla implementace algoritmu enumeration sort na lineárním poli procesorů, který byl prezentován během přednášek. Běh a kompilace programu je zprostředkován pomocí skriptu \texttt{test.sh}. Implementace využívá knihovny \textit{OpenMPI}.


\section{Rozbor a analýza algoritmu}

Enumeration sort je algoritmus pro seřazení všech prvků v poli pomocí nalezení konečné pozice všech prvků v poli. Toho algoritmus docílí pomoci porovnání všech prvků navzájem a určením kolik ostatních prvků je menších než daný prvek.

Algoritmus byl modifikován tak, aby dokázal řadit i duplicitní prvky. Pořadí u duplicitních prvků algoritmus určí dle indexu porovnávaných identických prvků.


\section{Implementace}

\section{Experimentální ověření časové složitosti}

\section{Komunikační protokol}


\section{Závěr}


\end{document}

