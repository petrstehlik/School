\documentclass[11pt,a4paper]{article}
\usepackage[left=2cm,text={17cm,24cm},top=3cm]{geometry}
\usepackage[T1]{fontenc}
\usepackage[czech]{babel}
\usepackage[utf8]{inputenc}
\usepackage{url}
\usepackage{graphicx}
\usepackage{pdfpages}
\graphicspath{ {img/} }

\begin{document}

\begin{center}
	\LARGE{Bezpečnost informačních systémů -- projekt 1}\\
	\large{Vysoké učení technické v Brně}
	\vspace{0.5cm}

	Petr Stehlík <xstehl14@stud.fit.vutbr.cz>

	\vspace{0.2cm}

	\today

\end{center}

\section{Mapování serverů}

Po připojení na server bis.fit.vutbr.cz jsem prvně zmapoval celou lokální síť, která sídlí v rozsahu 192.168.122.0/24, což jsem zjistil z nástroje ifconfig. Síť jsem mapoval pomocí dostupného nástroje nmap s následnými parametry: \texttt{nmap -v -sn 192.168.122.0/24}.

Z výsledků jsem následně zjistil, že je připojeno velké množství studentů, ty jsem odfiltroval z výsledku a našel jsem celkem 4 servery ptest1 -- ptest4, které jsem následně analyzoval.

Provedl jsem mapování portů a služeb, co na nich beží.

\subsection{ptest1}

\texttt{\$ nmap -p- ptest1\\
PORT     STATE SERVICE\\
22/tcp   open  ssh\\
80/tcp   open  http\\
8080/tcp open  http-proxy\\
}

\subsection{ptest2}
\texttt{\$ nmap -p- ptest2\\
PORT     STATE SERVICE\\
21/tcp   open  ftp\\
22/tcp   open  ssh\\
80/tcp   open  http\\
3306/tcp open  mysql\\
}

\subsection{ptest3}
\texttt{\$ nmap -p- ptest3\\
PORT   STATE SERVICE\\
22/tcp open  ssh\\
23/tcp open  telnet\\
}

\subsection{ptest4}
\texttt{\$ nmap -p- ptest4\\
PORT      STATE  SERVICE\\
22/tcp    open   ssh\\
80/tcp    open   http\\
41326/tcp closed unknown\\
41327/tcp closed unknown\\
41328/tcp closed unknown\\
41329/tcp closed unknown\\
41330/tcp closed unknown\\
41331/tcp closed unknown\\
41332/tcp closed unknown\\
41333/tcp closed unknown\\
41334/tcp closed unknown\\
41335/tcp closed unknown\\
41336/tcp closed unknown\\
41337/tcp open   unknown\\
}

\section{Získání jednotlivých tajemství}

Tajemství jsem získával náhodně dle postupného prohledávání a zkoumání serverů. Zde je uvedu v abecedním pořadí.

\subsection{Tajemství A}
Toto tajemství je umístěno na serveru ptest1 a je získatelné přes HTTP proxy na portu 8080. Zde je umístěn velmi jednoduchý přihlašovací formulář. Analyzoval jsem HTTP hlavičky odpovědi serveru a zjistil jsem, že server při neúspěšném přihlášení nastaví cookie \texttt{LOGGED_IN} na hodnotu \texttt{False}. Tuto cookie jsem upravil a nastavil její hodnotu na \texttt{True}. Poté stačilo znova poslat požadavek na server a byl jsem přihlášen. Na dané stránce získal tajemství.

\subsection{Tajemství B}
Tajemství B je též umístěno na serveru ptest1. Po prozkoumání internetové stránky na portu 80 jsem si všiml, že se vždy přesměruje na adresu ptest1/xsmith07. Pročetl jsem celý obsah stránek a usoudil, že se zkusím připojit přes SSH na ptest1 pomocí uživatele xsmith07. Jedno z nejpoužívanějších hesel je jméno domácího mazlíčka, které je zmíněno hned na hlavní stránce (Micák). Toto heslo jsem zkusil v několika kombinacích a heslo jsem uhádl. Tajemství B je přímo v domovském adresáři uživatele xsmith07.

\subsection{Tajemství C}



\end{document}
