\documentclass[11pt,a4paper]{article}
\usepackage[left=2cm,text={17cm,25cm},top=2.5cm]{geometry}
\usepackage[T1]{fontenc}
\usepackage[english]{babel}
\usepackage[utf8]{inputenc}
\usepackage{url}
\usepackage{graphicx}
\usepackage{pdfpages}
\usepackage[colorinlistoftodos,prependcaption,textsize=tiny]{todonotes}

\graphicspath{ {figs/} }

\begin{document}

\begin{center}
	\LARGE{Soft Computing}\\
	\Large{Job Performance Evaluation Using Back-propagation Network}
	\vspace{0.5cm}

    \begin{centering}
    \small{
        Bc. Petr Stehlík <xstehl14@stud.fit.vutbr.cz>
        }
    \end{centering}

	\vspace{0.2cm}

\end{center}

\section{Introduction}
Every user of a supercomputer needs to know whether their submitted job finished successfully and performed well. So far these tedious tasks are usually performed manually using only the output of their program and over-simplified metrics such as job run time and utilizied resources.

The aim of this project is to create a back-propagation neural network which can classify a job run whether it run well or it was in some way suspicious of unwanted behaviour such as poor performance or execution failure.

The network itself is supplied with fine-granular metric data acquired via Examon framework\cite{examon} which was run on Galileo supercomputer located in CINECA, Bologna, Italy. Where all job and metric data were gathered.

The structure of this document is as follows: in section \ref{sec:theory} the theoretical background needed for this project is presented together with the description of Examon framework. In the following section \ref{sec:data} the data supplied to the network are described as well as the final format of the data. Afterwards in section \ref{sec:implementation} the implementation of the network and its structure are laid out and in the last two sections \ref{sec:res} and \ref{sec:sum} the achieved results, summary and further work are discussed.

\section{Theoretical Background}
\label{sec:theory}

\subsection{Backpropagation Network}
Backpropagation networks are multi-layer feed-forward networks with supervised learning. There is no interconnection between neurons in the same layer but layers are fully connected to each other in order to be able to do forward and backward propagation.

Each neuron disposes of a weight for each input initialized to a random value in range $<0,1>$, activation function and transfer function. The input neurons 
\subsubsection{Feedforward Propagation}
\subsubsection{Backward Propagation}


\subsection{Examon}
\label{sec:examon}
Examon framework is used for exascale monitoring of supercomputing facilities. It is built on top of MQTT protocol\cite{locke2010mqtt} which allows measured metrics to be send to a central broker where received data are processed and stored in KairosDB\cite{KAIROS} database utilizing Cassandra\cite{CASSANDRA} cluster.

\begin{figure}[ht]
    \centering
    \includegraphics[width=0.5\linewidth]{examon-architecture}
    \caption{Examon framework architecture}
    \label{arch}
\end{figure}

KairosDB is used for storing metric data in time-series format whereas Cassandra, serving as a backend for KairosDB is also used for storing job-related data. More on data semantics is described in \ref{sec:data}.



\section{Data}
\label{sec:data}

As previously stated in \ref{sec:examon} data are gather via Examon framework and stored in Cassandra database. We can split the data into two categories. Job data and metric data.

The job data come from 


\section{Implementation}
In this section we describe the implementation phase of the project. The project was implemented using Python 2.7.13. External library SciPy\cite{scipy} was used for linear data interpolation.

\label{sec:implementation}

\subsection{Data Acquisition}
\label{sec:data-filter}
First, job data were queried and filtered according to several rules:
\begin{itemize}
    \item job run time must be between 10 and 60 minutes
    \item job must occupy the whole node (multiplies of 16 cores)
    \item job must be run in 15-day period
\end{itemize}

\subsection{Data Labelling}

\subsection{Metric Networks}

\subsection{Job Network}



\section{Achieved Results}
\label{sec:res}

\section{Summary}
\label{sec:sum}

\bibliography{doc}{}
\bibliographystyle{abbrv}

\end{document}
