\documentclass[11pt,a4paper]{article}
\usepackage[left=2cm,text={17cm,25cm},top=2.5cm]{geometry}
\usepackage[T1]{fontenc}
\usepackage[english]{babel}
\usepackage[utf8]{inputenc}
\usepackage{url}
\usepackage{graphicx}
\usepackage{pdfpages}
\usepackage[colorinlistoftodos,prependcaption,textsize=tiny]{todonotes}

\graphicspath{ {figs/} }

\begin{document}

\begin{center}
	\LARGE{Soft Computing}\\
	\Large{Job Performance Evaluation Using Back-propagation Network}
	\vspace{0.5cm}

    \begin{centering}
    \small{
        Bc. Petr Stehlík <xstehl14@stud.fit.vutbr.cz>
        }
    \end{centering}

	\vspace{0.2cm}

\end{center}

\section{Assignment}
The aim of the project is to assemble a mechanical device and create software for acquiring 2D hand geometry using a line camera.
A solution for camera mount and hand placement is proposed in the following sections for best geometry acquisition and image reconstruction.

\section{Theoretical Background}
\label{sec:theory}

\subsection{Backpropagation Network}
Backpropagation networks are multi-layer feed-forward networks with supervised learning. There is no interconnection between neurons in the same layer but layers are fully connected to each other in order to be able to do forward and backward propagation.

Each neuron disposes of a weight for each input initialized to a random value in range $<0,1>$, activation function and transfer function. The input neurons 
\subsubsection{Feedforward Propagation}
\subsubsection{Backward Propagation}


\subsection{Examon}
\label{sec:examon}
Examon framework is used for exascale monitoring of supercomputing facilities. It is built on top of MQTT protocol\cite{locke2010mqtt} which allows measured metrics to be send to a central broker where received data are processed and stored in KairosDB\cite{KAIROS} database utilizing Cassandra\cite{CASSANDRA} cluster.

\begin{figure}[ht]
    \centering
    \includegraphics[width=0.5\linewidth]{examon-architecture}
    \caption{Examon framework architecture}
    \label{arch}
\end{figure}

KairosDB is used for storing metric data in time-series format whereas Cassandra, serving as a backend for KairosDB is also used for storing job-related data. More on data semantics is described in \ref{sec:data}.



\section{Data}
\label{sec:data}

As previously stated in \ref{sec:examon} data are gather via Examon framework and stored in Cassandra database. We can split the data into two categories. Job data and metric data.

The job data come from 


\section{Implementation}
In this section we describe the implementation phase of the project. The project was implemented using Python 2.7.13. External library SciPy\cite{scipy} was used for linear data interpolation.

\label{sec:implementation}

\subsection{Data Acquisition}
\label{sec:data-filter}
First, job data were queried and filtered according to several rules:
\begin{itemize}
    \item job run time must be between 10 and 60 minutes
    \item job must occupy the whole node (multiplies of 16 cores)
    \item job must be run in 15-day period
\end{itemize}

\subsection{Data Labelling}

\subsection{Metric Networks}

\subsection{Job Network}



\section{Achieved Results}
\label{sec:res}
Solution described in this report was assembled and implemented with excellent results. Integration of all hardware parts and software control results
in images being scanned quickly in resolution exceeding 60 MPx. During testing it was proved to be effective to not only scan images
of hand geometry but also fingerprints which leads to general purpose sensoric solution for capturing multiple biometric data. Furthermore,
the solution is able to function with very limited resources necessary and provides instant access to scanned images via a web serber. Complete scan of both of the subject's
hands can be done under 1 minute.

\section{Summary}
\label{sec:sum}

\end{document}
