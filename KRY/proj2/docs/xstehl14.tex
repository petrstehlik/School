\documentclass[11pt,a4paper]{article}
\usepackage[left=2cm,text={17cm,25cm},top=2cm]{geometry}
\usepackage[T1]{fontenc}
\usepackage[czech]{babel}
\usepackage[utf8]{inputenc}
\usepackage{url}
\usepackage{graphicx}
\usepackage{pdfpages}
\usepackage{algorithmicx}
\usepackage{amsmath}
\usepackage{listings}

\begin{document}

\begin{center}
	\LARGE{Kryptografie -- dokumentace k projektu 2}\\
	\large{Vysoké učení technické v Brně}
	\vspace{0.2cm}

	Petr Stehlík <xstehl14@stud.fit.vutbr.cz>     \today

\end{center}

\section{Zadání}

Cílem projektu bylo vytvořit program pro generování parametrů pro algoritmus RSA, šifrování a dešifrování zprávy.
Program také implemetuje prolomení RSA pomocí faktorizace slabého veřejného modulu a jakéhokoliv složeného čísla do 96 bitů.

\section{Implementace generování parametrů}

Algoritmus generování parametrů $P, Q, N, E$ a $D$ pro RSA pracuje následovně:

\begin{itemize}
    \item Generuj dvě velká prvočísla $P$ a $Q$
    \item $N = P * Q$
    \item $\phi(n) = (P - 1) * (Q - 1)$
    \item Zvol náhodně $E$ mezi $1$ a $\phi(N)$ tak, že $gcd(e, \phi(N)) = 1$
    \item Vypočítej $D = inv(E, \phi(N))$ - $inv$ je operace nalezení inverzního prvku
    \item Veřejný klíč je dvojice $(E, N)$
    \item Soukromý klíč je dvojice $(D, N)$
\end{itemize}

Pro generování korektních parametrů $P$, $Q$ a $N$ platí, že pokud požadujeme $N$ o velikosti $n$ bitů, musí být parametry
$P$ a $Q$ z intervalu $(\sqrt[]2 \times 2^{n/2 - 1}, 2^{n/2})$\footnote{\url{https://bit.ly/2KHkVVi}}.

Pro generování čísel v daném rozsahu byla použita metoda třídy \texttt{gmp\_randclass.get\_z\_range} z knihovny GMP
následujícím způsobem: \texttt{get\_z\_range(end - start) + start}. Kde \texttt{start} a \texttt{end} označují začátek,
resp. konec intervalu.

Tyto čísla následně byla zkontrolována metodou Miller-Rabin zda jsou prvočísla. Pokud nebyly, generovaly se nová čísla
dokud nebylo generované číslo prvočíslem.

Následně se vypočítalo $\phi$ a $E$, kde pomocí euklidovy metody hledání největšího společného
dělitele\footnote{\url{https://en.wikipedia.org/wiki/Euclidean_algorithm}} se generoval vhodný parametr $E$.

Parametr $D$ se získal operací nalezení inverzního prvku rozšířeným euklidovským algoritmem\footnote{
\url{https://en.wikipedia.org/wiki/Extended_Euclidean_algorithm#Modular_integers}}.

Tímto postupem byly získány všechny potřebné parametry pro šifrování a dešifrování pomocí algoritmu RSA, které jsou
v hexadecimální podobě vypsání na standardní výstup oddělené mezerou v daném pořadí: $P, Q, N, E, D$.

\section{Implementace šifrování a dešifrování}

Pro šifrování algoritmem RSA je nutná dvojice $(E, N)$ a zpráva $M$. Šifrování probíhá následovně:
\begin{center}
    $C \equiv M^E \mod N$.
\end{center}

\noindent Dešifrovací algoritmus je následující:
\begin{center}
    $M \equiv C^D \mod N$.
\end{center}

\noindent V obou algoritmech je $C$ označením zašifrované a $M$ nezašifrované zprávy. Pro odstranění side-channel útoků
byla použita bezpečná funkce \texttt{mpz\_powm\_sec}\footnote{\url{https://gmplib.org/manual/Integer-Exponentiation.html}}.

\section{Implementace faktorizace}
Faktorizace je tvořena dvěma algoritmy. První, naivní, algoritmus vyzkouší všechna lichá čísla do $1000000$ zda nejsou
celočíselným dělitelem daného čísla.

Pro komplexnější řešení byl zvolen algoritmus Pollard Rho\footnote{\url{https://en.wikipedia.org/wiki/Pollard's_rho_algorithm}}
kvůli své efektivitě a jednoduchosti implementace.
Tento algoritmus již využívá teorie čísel a ze vztahu $N = P \times Q$ hledá faktor $P$. Dále disponuje funkcí
$g(x) = (x^2 + 1) \mod N$, která generuje pseudo-náhodnou posloupnost čísel. Tato posloupnost je konečná a tudíž je
nutné detekovat zacyklení. To je kontrolováno pomocí Floydova algoritmu, který nalezne největšího společného dělitele
absolutního rozdílu dvou po sobě následujících pseudonáhodných $X$ a čísla $N$. Jakmile je tento faktor větší než $1$,
našli jsme hledané číslo $P$.

Hledaný faktor je následně vypsán jako hexadecimální číslo na standardní výstup a program je ukončen.

\section{Závěr}

Program úspěšně implemetuje generování parametrů pro algoritmus RSA, šifrování a dešifrování zprávy a faktorizaci parametru $N$.
Program pro testované vstupy vždy faktorizoval 96bitová čísla do 120 sekund od spuštění programu.

\end{document}

