\documentclass[11pt,a4paper]{article}
\usepackage[left=2cm,text={17cm,25cm},top=2.5cm]{geometry}
\usepackage[T1]{fontenc}
\usepackage[czech]{babel}
\usepackage[utf8]{inputenc}
\usepackage{url}
\usepackage{graphicx}
\usepackage{pdfpages}
\graphicspath{ {img/} }

\begin{document}

\begin{center}
	\LARGE{Kódování a komprese dat -- projekt 3 GIF2BMP}\\
	\large{Vysoké učení technické v Brně}
	\vspace{0.5cm}

	Petr Stehlík <xstehl14@stud.fit.vutbr.cz>

	\vspace{0.5cm}

	\today

\end{center}

\section{Zadání}
Cílem projektu bylo vytvořit knihovnu pro převod souboru grafického formátu GIF na soubor grafického formátu BMP (Microsoft Windows Bitmap). S pomocí této knihovny byl dále implementovaný konzolový program gif2bmp, který převod souborů provede.

Knihovna je implementována pro konverzi statického GIF89a na BMP bez komprese.

\section{Implementace}
Knihovna je implementována v jazyku C++ s použitím standardních knihoven pro C/C++ ve standardu C++11. Knihovna zpřístupňuje jedinou funkci \texttt{gif2bmp} sloužící pro daný převod. Implementace daného rozhraní využívá třídy \texttt{GIF} a \texttt{BMP}, které implementují práci se soubory v grafickém formátu GIF a BMP.

\subsection{\texttt{gif2bmp} aplikace}

Konzolová aplikace zpracováva celkem 4 parametry:

\begin{itemize}
	\item{-i <input file> cesta ke vstupnímu souboru, pokud parametr není zadán, je vstupním souborem stadardní vstup,}
	\item{-o <output file> cesta k výtupnímu souboru, pokud parametr není zadán, je výtupním souborem stadardní výstup,}
	\item{-l <log file> cesta k souboru s výstupní zprávou, pokud parametr není zadán, je výpis zprávy ignorován,}
	\item{-h vypíše nápovědu a program se ukončí.}
\end{itemize}

parsovani argumentu, spusteni gif2bmp a výpis info o prevodu

\subsection{Třída \texttt{GIF}}

\subsection{Třída \texttt{BMP}}

Třída \texttt{BMP} implementuje operace nutné pro vytvoření a uložení grafického formátu BMP. Třída vnitřně definuje hlavičkové struktury formátu BMP, které jsou používány při vytvoření jednotlivých souborů. Konstruktor třídy přijímá dva parametry -- šířku a výšku tvořeného obrázku. První věřejně dostupnou metodou je metoda \texttt{Store}, která přijímá file pointer na soubor, kam zapíše data z druhého parametru. Druhý parametr je typu \texttt{vector<color>}, kde \texttt{color} je struktura definovaná v třídě \texttt{GIF}.




\end{document}
